Note that taking $t = (0, 0, \dotsc, 1, \dotsc, 0, 0)$ with a $1$ in the $i$\textsuperscript{th} position, we have that
\[
\var \hat{\beta}_i \leq \var \beta_i^*
\]

\begin{definition}
  We have the following values
\[
\hat{\vec{Y}} = X \hat{\vec{\beta}} \text{ is the vector of fitted values\index{fitted values}.}
\]
\[
\vec{R} = \vec{Y} - \vec{\hat{Y}} \text{ is the vector of residuals\index{residuals}.}
\]
Then we define the residual sum of squares\index{residual sum of squares} (RSS) as
\[
RSS = \norm{R}^2 = (\vec{Y} - X \vec{\hat{\beta}})^T(\vec{Y} - X \vec{\hat{\beta}} \mpunct{.}
\]
\end{definition}

Note that $X^T \vec{R} = X^T Y - X^T X \hat{\vec{\beta}} = 0$ by \eqref{eq:3.5}.
Hence $\vec{R}$ is orthogonal to the column space of $X$, and $\hat{\vec{Y}}$ is just the orthogonal projection of $\vec{Y}$ onto the column space of $X$.
Write $\hat{\vec{Y}}$ as follows
\[
\hat{\vec{Y}} = X \vec{\hat{\beta}} = X(X^TX)^{-1}X^T\vec{Y} = P \vec{Y}
\]
where $P = X(X^TX)^{-1}X^T$. $P$ represents an orthogonal projection of $\RR^n$ onto the column space of $X$.
Note that we have $P^2 = P$ (i.e. $P$ is idempotent) and $P^T = P$ (i.e. $P$ is symmetric).

\section{Linear models with normal theory assumptions}
\label{sec:3.5}
Suppose that we have a model $\vec{Y} = X \vec{\beta} + \vec{\epsilon}$, with $\vec{\epsilon} \sim \Normal_n(0, \sigma^2I)$ and $\mathop{rank} X = p ( < n)$.
Note that this is a special case of \cref{sec:3.4}, hence all previous results hold.
In particular, we have that $\vec{Y} \sim \Normal_n (X \vec{\beta}, \sigma^2 I)$.

The log-likelihood is
\[
\ell ( \vec{\beta}, \sigma^2 ) = -\frac{n}{2} \log (2 \pi) - \frac{n}{2} \log \sigma^2 - \frac{1}{2 \sigma^2} S(\vec{\beta})
\]
where $S(\beta) = (\vec{Y} - X \vec{\beta})^T(\vec{Y} - X \vec{\beta})$.
Maximizing $\ell$ with respect to $\vec{\beta}$ is equivalent to minimizing $S(\vec{\beta})$ is $\hat{\vec{\beta}} = (X^TX)^{-1}X^T\vec{Y}$, i.e. the same as the least square estimator.
For the maximum likelihood estimator of $\sigma^2$, we require that
\[
\left. \frac{\partial \ell}{\partial \sigma^2} \right\vert_{(\hat{\vec{\beta}}, \hat{\sigma^2})} = 0 \mpunct{,}
\]
\[
\text{i.e. } -\frac{n}{2 \hat{\sigma}^2} + \frac{1}{\hat{\sigma}^4} = 0
\]
\[
\text{i.e. } \hat{\sigma}^2 = \frac{1}{n} S (\hat{\vec{\beta}}) = \frac{1}{n}(\vec{Y} - X \hat{\vec{\beta}})^T(\vec{Y} - X \hat{\vec{\beta}})
\]
\[
\text{i.e. } \hat{\sigma}^2 = \frac{1}{n}\mathrm{RSS} \mpunct{.}
\]
See the example sheet for $\hat{\vec{\beta}}$ and $\hat{\sigma}^2$ for simple linear regression and one way anova (analysis of variance).

\begin{lemma}[label=lemma:3.5]
  If $\vec{Z} \sim \Normal_n (0, \sigma^2 I)$ and $A$ is $n \times n$ symmetric and idempotent with rank $r$, then $\vec{Z}^T A \vec{Z} \sim \sigma^2 \chi^2_r$.
\end{lemma}

\begin{proof}
  As $A$ is idempotent, we have that all eigenvalues $\lambda_i$ verify $\lambda_i \in \{ 0, 1 \}$ for $i = 1, \dotsc, n$.
As $A$ is symmetric, there exists an orthogonal matrix $Q$ such that
\begin{equation}
  \label{eq:lemma:3.5}
Q^TAQ = \mathop{diag}(\lambda_1, \dotsc, \lambda_n) = \mathop{diag} (\underbrace{1, \dotsc, 1}_{r}, \underbrace{0, \dotsc, 0}_{n - r}) = \Lambda \mpunct{.}
\end{equation}
Now let $\vec{W} = Q^T\vec{Z}$, so $\vec{W} \sim \Normal_n( 0, \sigma^2 I)$, so we have that
\begin{IEEEeqnarray*}{rCl}
  \vec{Z}^T A \vec{Z} &=& \vec{W}^T Q^T A Q \vec{W} \\
&=& \vec{W}^T \Lambda \vec{W} \\
&=& \sum_{i=1}^r W_i^2 \sim \sigma^2 \chi^2 \mpunct{.}
\end{IEEEeqnarray*}
\end{proof}

\begin{lemma}[label=lemma:3.6]
  For a symmetric idempotent matrix $A$, we have that $\mathop{rank} A = \tr A$
\end{lemma}
\begin{proof}
From \eqref{eq:lemma:3.5} we have that
\begin{IEEEeqnarray*}{rCl}
 \mathop{rank} A &=& \mathop{rank} \Lambda \\
&=& \tr Q^T A Q \\
&=& \tr A Q Q^T \\
&=& \tr A \mpunct{.}
\end{IEEEeqnarray*}
\end{proof}

\begin{theorem}[label=thm:3.7, name=joint distribution of $\vec{\hat{\beta}}$ and $\hat{\sigma}^2$]
  We have the following:
  \begin{enumerate}
  \item $\vec{\hat{\beta}} \sim \Normal_p \left(\vec{\beta}, \sigma^2 (X^TX)^{-1}\right)$.
  \item  $\mathrm{RSS} \sim \sigma^2\chi^2_{n - p}$ and so $\hat{\sigma}^2 \sim \frac{\sigma^2}{n} \chi^2_{n - p}$
  \item $\vec{\hat{\beta}}$ and $\hat{\sigma}^2$ are independent.
  \end{enumerate}
\end{theorem}

\begin{proof}
  We have that
  \begin{enumerate}
  \item $\vec{\hat{\beta}} = (X^TX)^{-1}X^T\vec{Y} \ (= C\vec{Y})$ say, hence we have that
\[
\vec{\hat{\beta}} \sim \Normal_n \left(\vec{\beta}, \sigma^2 (X^TX)^{-1}\right) \mpunct{.}
\]
\item We apply \cref{lemma:3.5} to $\vec{Z} = \vec{Y} - X\vec{\beta} \sim \Normal_n (0, \sigma^2 I)$, and $A = I - P$ where $P = X(X^TX)^{-1}X^T$.
$P$ is symmetric and idempotent, hence $A = I - P$ is symmetric and idempotent.
By \cref{lemma:3.6}, we have that
\begin{IEEEeqnarray*}{rCl}
  \mathop{rank} P &=& \tr P \\
&=& \tr X(X^TX)^{-1}X^T \\
&=& \tr (X^TX)^{-1}X^TX \\
&=& \tr I_p \\
&=& p \mpunct{.}
\end{IEEEeqnarray*}
hence we have that $\mathop{rank} P = \mathop{rank} (I - P) = n - p$.
By \cref{lemma:3.5}, we have that $\vec{Z}^TA\vec{Z} \sim \sigma^2 \chi^2_{n-p}$.
Note that $(I - P)X = 0$, so that
\begin{IEEEeqnarray*}{rCl}
  \vec{Z}^TA\vec{Z} &=& (\vec{Y} - X\vec{\beta})^T(I - P)(\vec{Y} - X \vec{\beta}) \\
&=& \vec{Y}^T(I - P)\vec{Y} \mpunct{.}
\end{IEEEeqnarray*}
We know that $\vec{R} = \vec{Y} - \hat{\vec{Y}} = (I - P)\vec{Y}$, so we have that
\[
\mathrm{RSS} = \vec{R}^T\vec{R} = \vec{Y}^T(I - P)^T(I - P)\vec{Y} = \vec{Y}^T(I - P)\vec{Y} \mpunct{,}
\]
so we have that
\[
\mathrm{RSS} = \vec{Z}^TA\vec{Z} \sim \sigma^2 \chi^2_{n-p}
\]
\item Consider the vector $\vec{V} = D\vec{Y}$ where
\[
\vec{V} \begin{pmatrix}\hat{\vec{\beta}} \\ \vec{R}\end{pmatrix} \text{ and } D = \begin{pmatrix}C \\ I - P\end{pmatrix} \mpunct{.}
\]
By \cref{prop:3.1}, we have that $\vec{V}$ is multivariate normal with
\begin{IEEEeqnarray*}{rCl}
\cov \vec{V} &=& \sigma^2 D D^T \\
 &=& \sigma^2 \begin{pmatrix}C \\ I - P\end{pmatrix}\left( C^T (I - P)^T \right) \\
&=& \sigma^2 \begin{pmatrix}C C^T & C (I - P) \\ (I - P)C^T & I - P\end{pmatrix} \mpunct{.}
\end{IEEEeqnarray*}
we have that $C(I-P) = 0$ as $(X^TX)^{-1}X^T(I - P) = 0$.
Hence, we have that $\hat{\vec{\beta}}$ and $\vec{R}$ are independent by \cref{prop:3.2}.
Thus, we have that $\hat{\vec{\beta}}$ and $\mathrm{RSS} = \vec{R}^T\vec{R}$ are independent, and so $\hat{\vec{\beta}}$ and $\hat{\sigma^2}$ are independent.
From (ii), we know that $\EE[\mathrm{RSS}] = \sigma^2/(n-p)$, so we have that
\[
\tilde{\sigma^2} = \frac{\mathrm{RSS}}{n - ^p}
\]
is an unbiased estimator for $\sigma^2$.
  \end{enumerate}
\end{proof}

%%% Local Variables:
%%% mode: latex
%%% TeX-master: "statistics"
%%% End:
