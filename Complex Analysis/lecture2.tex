\section{Conformal mappings}

\begin{definition}
  $f$ is conformal\index{conformal} at $w$ if $f$ is holomorphic at $w$ and $f'(w) \neq 0$.
\end{definition}

Note that if $f$ is conformal at $w$, then by the inverse function theorem, $f$ is locally bijective, and the local inverse function is conformal.

Two open sets $U$, $V$, related by a conformal bijection are said to be conformally equivalent\index{conformally equivalent}.

Viewing $f$ as a map $R^2 \rightarrow R^2$, then $f$ is locally invertible if $\det Df \neq 0$. Here, we have that $\det Df = u_xv_y - v_xu_y = u_x^2 + u_y^2$ (by the Cauchy-Riemann equations), hence we have that $f'(z) \neq 0$ implies that $\det Df > 0$. Note that this means that the map also preserves orientation.

Conformal maps preserve angles. Let $f$ be holomorphic on $U$. At $w \in U$, take two paths:
\[
g_i : [-1 ; 1] \rightarrow U, \gamma_i(0) = w, \gamma_i \text { differentiable (at 0). }
\]
Then we define the angle between $\gamma_1$ and $\gamma_2$ as:
\[
\text{angle}\left(\gamma_1, \gamma_2\right) = \arg \left(\gamma_1'(0)\right) - \arg \left(\gamma_2'(0)\right) \mpunct{.}
\]
Then, the paths are transformed under $f$ to $f \circ \gamma_i : [-1, 1] \rightarrow \CC$, and the angle becomes:
\[
\text{angle}\left(f \circ \gamma_1, f \circ \gamma_2\right) = \arg \left( (f \circ \gamma_1)'(0)\right) - \arg \left( (f \circ \gamma_2)'(0)\right) \mpunct{.}
\]
If $f$ is conformal at $w$, we have that:
\[
(f \circ \gamma_i)'(0) = f'\left(\gamma_i(0)\right)\gamma_i'(0) = f'(w)\gamma_i'(0)
\]
hence we can deduce that:
\[
\text{angle}\left(f \circ \gamma_1, f \circ \gamma_2\right) = \arg\left(\frac{\gamma_1'(0)}{\gamma_2'(0)}\right) = \text{angle}\left(\gamma_1, \gamma_2\right) \mpunct{.}
\]
Conformal maps preserve angles.

\paragraph{Example}

\begin{itemize}
\item Möbius maps $z \mapsto \frac{az+b}{cz+d}$, $ad - bc \neq 0$, from $\CC \cup \{ \infty \} \rightarrow \CC \cup \{ \infty \}$ are invertible, hence everywhere conformal.
\item The map $z \rightarrow z^n$ is everywhere homomorphic, and conformal except at $0$. This defines a conformal equivalence $\{ z \in \CC^* \mid 0 < \arg z < \pi/n \} \rightarrow \{ z \in \CC \mid Im z > 0 \}$.
\item The upper half plane is conformally equivalent to the open unit disk. Indeed, we have that $z \in H \Leftrightarrow \abs{z - i} < \abs{z + i} \Leftrightarrow \abs*{\frac{z-i}{z+i}} < 1$ (where $H$ denotes the upper half plane). So $z \rightarrow \frac{z-i}{z+i}$ takes $\{ \Im z > 0 \} \rightarrow \{ \abs{w} < 1 \}$. Note that since $f$ is the restriction of a Möbius map it is conformal.
\item The Jouhowski transformation\index{Jouhowski transformation}. Consider the map $z \mapsto w = \frac{1}{2}\left(z + \frac{1}{z}\right)$ (it can be proved that $\frac{w + 1}{w - 1} = \left(\frac{z + 1}{z - 1}\right)^2$). 
The map $f$ is holomorphic except at $0$, and the map is conformal (in $\CC^*$) except at $z = \pm 1$. 
In fact, we have that $f'(z) = 1 - \frac{z^2 + 1}{2z^2}$. If $z = re^{i\theta}$, and $w = u + iv$, $r, \theta, i, v \in \RR$, then we have:
\[
u = \frac{1}{2}\left(r + \frac{1}{r}\right)\cos \theta \text{ and } v = \frac{1}{2}\left(r + \frac{1}{r}\right)\sin \theta \mpunct{,}
\]
and hence a circle centered at the origin is mapped onto an ellipse:
\[
\{ \abs{z} = \rho \} \mapsto^f \left\{ \frac{u^2}{\frac{1}{4}\left(\rho + \frac{1}{\rho^2}\right)^2} + \frac{v^2}{\frac{1}{4}\left(\rho - \frac{1}{\rho}\right)^2} = 1 \right\} \mpunct{.}
\]
Now consider an off-centre circle through $-1$ and $-i$, then the image looks as follows [insert graph here], which is a crude approximation to an aerofoil. The Jouhowksi transformation can be used to transform fluid flow over a wing to understanding flow across a cylinder.
\end{itemize}

Observation: in building conformal equivalences by hand, it is often useful to describe a region bound by circular arcs as follows:
\[
\arg\left(\frac{z-\alpha}{z-\beta}\right) \in [\mu^+, \mu^-]
\]
\begin{figure}
  \centering
  \begin{tikzpicture}
    \tkzInit[xmin=-5,xmax=6]
    \tkzAxeX

    \tkzDefPoint(0,0){O}
    \tkzDefPoint(10,0){I}

    \tkzDefPoint(2, 2){C}
    \tkzDefShiftPoint[C](50:2){z}

    \tkzDefShiftPoint[C](10:2){alpha}
    \tkzDefShiftPoint[C](190:2){beta}

    \tkzDrawPoints(alpha,beta,z)


    \tkzDrawArc(C,alpha)(beta)
    \tkzDrawArc[style=dashed](C,beta)(alpha)

    \tkzInterLL(z,alpha)(O,I) \tkzGetPoint{iAlpha}
    \tkzInterLL(z,beta)(O,I) \tkzGetPoint{iBeta}

    \tkzDrawSegments(z,iAlpha z,iBeta)

    \tkzMarkAngle[fill=blue!25,mkpos=.2,size=0.5cm](I,iAlpha,alpha)
    \tkzMarkAngle[fill=green!25,mkpos=.2,size=0.5cm](O,iBeta,beta)
    \tkzMarkAngle[fill=red!25,mkpos=.2,size=0.5cm](beta,z,alpha)

    \tkzLabelPoint[above](z){$z$}
    \tkzLabelPoint[above right](alpha){$\alpha$}
    \tkzLabelPoint[left](beta){$\beta$}
    
    \tkzLabelAngle[pos=0.75](I,iAlpha,alpha){$\theta$}
    \tkzLabelAngle[pos=0.75](O,iBeta,beta){$\phi$}
    \tkzLabelAngle[pos=0.75](beta,z,alpha){$\mu$}
  \end{tikzpicture}
  \caption{Region bound by a circular arc}
  \label{fig:region_bound_by_circular_arc}
\end{figure}
As we can see in \cref{fig:region_bound_by_circular_arc}, as $z$ moves on the arc of circle, the angle $\mu$ is constant, and $\mu = \theta - \phi = \arg ( z - \alpha ) - \arg ( z - \beta )$.

\paragraph{Example}
The upper half of the open disk is conformally equivalent to the upper half plane $H$ and to the open unit disk. 
Consider the region $\{ z \mid \Im z > 0, \abs{z} < 1 \} = \{ z = \mid \pi/2 < \arg \left(\frac{z-1}{z+1}\right) < \pi\}$. Then the map $z \mapsto \frac{z-i}{z+1} = w$ takes $R$ to the upper left quadrant after which the map $w \mapsto -w^2$ maps the upper left quadrant to the upper half plane.

\paragraph{Riemann mapping theorem}
Let $D \subseteq \CC$ be any domain (open set) bound by a simple closed curve. Then there exisits a conformal bijection $\phi ! D \rightarrow \{ \abs{z} < 1 \}$. In particular, the interior of the Koch snowflake is conformally equivalent to the unit disk.



%%% Local Variables: 
%%% mode: latex
%%% TeX-master: "complex_analysis"
%%% End: 
