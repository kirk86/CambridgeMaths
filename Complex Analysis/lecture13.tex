
\paragraph{Example}

Let us compute the following integral:
\[
\int_0^\infty \frac{1}{1 + x^4} \: dx
\]

Let $\gamma_R$ be the closed contour comprising $[-R, R] \subseteq \RR$ and the arc of semicircle of radius $R$. $f(z) = 1/(1 + z^4)$ has poles at $e^{i \pi / 4}$ and $e^{3 i \pi /4}$ inside $\gamma$.

Hence we have that
\[
\int_{\gamma_R} \frac{1}{1 + z^4} \: dz = 2 \pi i \sum_{p_i} \mathrm{Residues}
\]
but we have that
\[
\int_{\gamma_R} \frac{1}{1 + z^4} \: dz = \int_{-R}^R \frac{1}{1 + x^4} \: dx + \int_0^\pi \frac{i R e^{i\theta}}{1 + R^4 e^{4 i \theta}} \: d\theta \mpunct{.}
\]

The integral over the semicircle has order $O(1/R^3)$, and hence vanishes as $R \rightarrow \infty$. Now we have that
\[
\Res_f(p_i) = \frac{1}{4 p_i^3}
\]
and since $p_i^4 = -1$, we have $\Res_f (p_i) = -p_i/4$, so we have that
\[
\int_0^\infty \frac{1}{1 + x^4} \: dx = \frac{\pi}{2 \sqrt{2}} \mpunct{.}
\]

\paragraph{Example}
Compute the following integral
\[
\int_{-\infty}^\infty \frac{\cos x}{1 + x + x^2} \: dx \mpunct{.}
\]
We know that $1/(1 + z + z^2)$ has a simple pole at $ê^{2 \pi i /3}$.
Note that $\cos z$ becomes large on a semicircle of radius $R$ (it takes value $O(e^R)$), thus we take instead
\[
f(z) = \frac{e^{i z}}{1 + z + z^2} \mpunct{.}
\]

Then, we have that, putting $\gamma_R$ the semicircle of radius $R$
\begin{IEEEeqnarray*}{rCl}
\int_{\gamma_R} f(z) dz &=& \int_{-R}^R f(x) dx + \int_0^\pi f(R e^{i\theta})i R e^{i\theta} d\theta \\
&=& 2 \pi i \Res_f (e^{2 \pi /3}) \\
&=& 2 \pi i \frac{e^{i w}}{2w + 1} \\
&=& \frac{2 \pi}{\sqrt{3}} (e^{i (-1 + \sqrt{3}i)/2}) \mpunct{.}
\end{IEEEeqnarray*}
where $w = e^{2 \pi i / 3}$.

Note that
\[
\Re \left(\int_{-R}^R f(x) dx \right) = \int_{-\infty}^\infty \frac{\cos x}{1 + x + x^2} dx \mpunct{.}
\]
We also have that
\[
\abs*{\int_0^\pi f(R e^{i\theta}) i R e^{i\theta} d\theta} \leq \int_0^\pi \frac{R e^{- R \sin \theta}}{\abs{1 + R e^{i \theta} + R^2 e^{2 i \theta}}} d\theta \mpunct{.}
\]

\paragraph{Example}
Compute the following integral
\[
\int_0^\infty \frac{\cos x}{x} dx \mpunct{.}
\]

Consider the contour $\gamma_{R}$, then we have that this contour runs through a pole of the integrand. Hence, we replace $\gamma_R$ by a modified contour $\gamma_{R, \epsilon}$.

\begin{lemma}
 Let $f$ be holomorphic on $B_a(r) \setminus \{ a \}$ with a simple pole at $a$.
 Let $0 < \epsilon < r$, and let $\gamma_\epsilon : [\alpha, \beta] \rightarrow \CC$ be $t \mapsto a + \epsilon e^{it}$.
 Then we have that
\[
\lim_{\epsilon \rightarrow 0} \int_{\gamma_\epsilon} f(z) dz = (\beta - \alpha)i \Res_f (a) \mpunct{.}
\]
\end{lemma}

\begin{proof}
  We write $f(z) = \frac{c}{z - a} + g(z)$ where $g$ is holomorphic, and $c = \Res_f (a)$.
  We have that
\[
\abs*{\int_{\gamma_\epsilon} g(z) dz} \leq (\beta - \alpha) \epsilon \sup_{\gamma_\epsilon} \abs{ g(z) } \mpunct{.}
\]
but $g$ is holomorphic, hence in particular continuous and bounded in a neighbourhood of $a$.
Thus, we have that this quantity tends to $0$ as $\epsilon$ tends to $0$.
Hence, we have that
\[
\lim_{\epsilon \rightarrow 0} \int_{\gamma_\epsilon} f(z) dz = \lim_{\epsilon \rightarrow 0} \int_{\gamma_\epsilon} \frac{c}{z - a} dz \mpunct{.}
\]
This is equal to
\[
c \lim_{\epsilon \rightarrow 0} \int_\alpha^\beta \frac{1}{\epsilon e^{i t}} \epsilon e^{i t} dt = i (\beta - \alpha) c \mpunct{.}
\]
\end{proof}

\begin{lemma}[Jordan's lemma]
 Let $f$ be holomorphic on $\{ \abs{z} > r \}$ and suppose that on this region, $z f(z)$ is bounded.
 Let $\gamma_R : [0, \pi] \rightarrow \CC$ be the path $t \mapsto R e^{i t}$.
 Then, for all $\alpha > 0$ we have that
\[
f(z) e^{i \alpha z} dz \rightarrow 0 \text{ as } R \rightarrow \infty \mpunct{.}
\]
\end{lemma}

\begin{proof}
  Suppose $\abs{f(z)} \leq C/\abs{z}$ for sufficiently large $\abs{z}$.
  On $(0, \pi/2]$, we have that
\[
\frac{d}{d\theta} \left(\frac{\sin \theta}{\theta}\right) = \frac{\theta \cos \theta - \sin \theta}{\theta^2} \leq 0 \mpunct{.}
\]
Hence $\sin \theta / \theta$ is decreasing, and $\sin t \geq 2 t/\pi$ for $t \in [0, \pi/2]$.
On $\gamma_R$ we have that $\abs{e^{i \alpha z} = e^{- \alpha R \sin t}$, but we have that
\[
e^{- \alpha R \sin t} \leq
\begin{cases}
  e^{-2 \alpha R t / \pi} & 0 \leq t \leq \pi/2  \mpunct{,}\\
  e^{-2 2 R t' /\pi} & 0 \leq t' = \pi - t \leq \pi /2 \mpunct{.}
\end{cases}
\]
then we have that
\[
\abs*{ \int_0^{\pi/2} e^{i \alpha R e^{it}} f(R e^{it}) i R e^{it} dt }
&\leq& \int_0^{\pi/2} e^{- \alpha R t} C dt
&=& \frac{1}{\alpha R} (1 - e^{- \alpha R \pi /2})
\]
\end{proof}
%%% Local Variables:
%%% mode: latex
%%% TeX-master: "complex_analysis"
%%% End:
