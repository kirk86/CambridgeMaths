\section{Taylor's theorem}

\begin{theorem}[name=Taylor's theorem, label=thm:taylor]
  Let $f : B_a(r) \rightarrow \CC$ be holomorphic. Then $f$ has a convergent power series representation on $B_a(r)$
\[
f(w) = \sum_{n=0}^\infty c_n(w - a)^n \mpunct{.}
\]
The coefficients are given by
\[
c_n = \frac{f^{(n)}(a)}{n!} = \frac{1}{2 \pi i} \int_{\partial B_a(\rho)} \frac{f(z)}{(z-a)^{n+1}} \: dz
\]
where $0 < \rho < r$.
\end{theorem}

\begin{remark}
We saw directly that a convergent power series was infinitely differentiable, and that the coefficients of power series are given by $c_n = f^{(n)}(a)/n!$.
\end{remark}

\begin{proof}
Consider a point $w$ with $\abs{w - a} < \rho < r$. By \cref{thm:cauchy}, we have that:
\[
f(w) = \frac{1}{2 \pi i}\int_{\partial B_a(\rho)} \frac{f(z)}{z - w} dz \mpunct{.}
\]
Now remark that we have
\[
\frac{1}{z - w} = \frac{1}{(z - a)\left( 1 - \frac{w - a}{z - a} \right)} = \sum_{n = 0}^\infty \frac{(w-a)^n}{(z-a)^{n+1}} \mpunct{.}
\]
By comparison to a geometric progression, this converges uniformly on $\abs{z - a} = \rho$.
As we have that the Riemann integral behaves nicely with respect to uniform convergence, we can exchange summation of a uniformly convergent power series and its integration.
Hence, we have that
\begin{IEEEeqnarray*}{rCl}
f(w) &=& \frac{1}{2 \pi i} \int_{\partial B_a(\rho)} f(z) \sum_{n=0}^\infty \frac{(w-a)^n}{(z- a)^{n+1}} dz \\
&=& \sum_{n=0}^\infty \left( \frac{1}{2 \pi i} \int_{\partial B_a(\rho)} \frac{f(z)}{(z-a)^{n+1}} \right) (w-a)^n \mpunct{.}
\end{IEEEeqnarray*}
which exhibits a convergent power series representing $f$.
\end{proof}

Contrast that result with the fact that  $exp \left(-\frac{1}{x^2}\right)$ has identically zero ``Taylor series'' at $0$. This gives a proof of Cauchy's formula for derivatives, but it is a bit sleight-of-hand. We sketch an alternative

\begin{proof}
  We want to prove that:
\[
f^{(n)}(a) = \frac{n!}{2 \pi i} \int_\gamma \frac{f(z)}{(z-a)^{n+1} dz \mpunct{.}
\]
  Proof by induction on $n$. Suppose that the result holds for $n = k$.
  We have that
  \begin{IEEEeqnarray*}{rCl}
f^{(k)}(a + h) - f^{(k)}(a) &=& \frac{k!}{2 \pi i} \int_{\partial B_a(2r)} f(w) \left( \frac{1}{(w - a -h)^{k+1}} - \frac{1}{(w-a)^{k+1}} \right) dw \\
&=& \frac{k!}{2 \pi i} \int_{\partial B_a(2r)} f(w) \left( (k+1) \int_{ [a, a+h] } (w - \xi)^{-(k+2)} d\xi \right) dw \mpunct{.}
  \end{IEEEeqnarray*}

Now consider
\[
F(h) = \frac{f^{(k)}(a+h) - f^{(k)}(a)}{h} - \frac{(k+1)!}{2 \pi i}\int_{\partial B_a(2r)} \frac{f(w)}{(w-a)^{k+2}} dw \mpunct{.}
\]
and use the fundamental theorem of calculus a second time to obtain
\[
F(h) = \frac{(k+2)!}{2 \pi i h} \int_{\partial B_a(2r)} f(w) \left( \int_{[a, a+h]} \int_{[a, \xi]} (w - \tau)^{-(k+3)} d\tau \: d\xi \right) \: dw \mpunct{.}
\]
Now we know that $f$ is holomorphic, hence continuous and bounded on $\overline{B_a(2r)}$, say $\abs{f(z)} < M$. We have that $\xi \in [a, a+h]$ implies $\abs{w - \tau} \leq r$ for $w \in \partial B_a(2r)$, and that $\tau \in [a, a+h]$ implies that
\[
\abs{F(h)} \leq \frac{(k+2)!}{2 \pi \abs{h}} 4 \pi r M \frac{abs{h}^2}{r^{h+3}} = O(\abs{h}) \mpunct{.}
\]
\end{proof}

\begin{corollary}[Morera's theorem]
Let $U$ be a domain and $f : U \rightarrow \CC$ continuous. If we have that
\[
\int_\gamma f(z) dz = 0
\]
for all closed paths $\gamma$, then $f$ is holomorphic on $U$.
\end{corollary}

Note that this is a ``converse'' to Cauchy's theorem bu we do not assume $U$ is simply connected.

\begin{proof}
As $\int_\gamma f(z) dz = 0$ for all closed paths $\gamma$, we know that $f$ has an antiderivative $F(z)$, defined as
\[

F(z) = \int_{a_0}^z f(w) dw \mpunct{.}
\]
with $F'(t) = f(t)$.
By definition, $F(z)$ is holomorphic, as it is clearly differentiable with derivative $f$.
So $F(z)$ is infinitely differentiable, and in particular, $f(z) = F'(z)$ is holomorphic.
\end{proof}

Let $f : B_a(r) \rightarrow \CC$ be holomorphic. We have $f(z) = \sum_{n \geq 0} c_n(z-a)^n$.
Observe that if $\forall n, c_n  = 0$ then $f \equiv 0$ on $B_a(r)$.
Otherwise, there exists a smallest $N > 0$ such that $c_N \neq 0$.
Then $f(z) = (z - a)^N g(z)$ where $g(z) = \sum_{n \geq N}c_n(z-a)^{n-N}$ and $g(a) \neq 0$.

\begin{definition}
  If $N > 0$, we say hat $f$ has a zero of order $N$ at $a$. We have that
\[
N = min \{ n \mid f^{(n)}(a) \neq 0 \}
\]
\end{definition}

\begin{lemma}[Principle of isolated zeros]
  If $f : B_a(r) \rightarrow \CC$ is holomorphic and $f$ is not identically $0$, then there exists some $0 \rho < r$ such that $f(z) \neq 0$ for $z \in B_a(\rho) \setminus \{ a \}$.
\end{lemma}

\begin{proof}
  If $f(a) \neq 0$, this is true by continuity of $f$.
  If there exists a zero of order $N$, then the result holds by continuity of $g$, where $f(z) = (z-a)^N g(z)$.
\end{proof}

%%% Local Variables:
%%% mode: latex
%%% TeX-master: "complex_analysis"
%%% End:
