\begin{definition} \label{def:5}
  Let $L/K$ be an extension, and $\alpha \in L$ be algebraic over $K$. As $K[X]$ is a principal ideal domain, we have that:
  \begin{equation*}
    I_\alpha = (P_\alpha) = \{ \text{multiples of } \alpha \}
  \end{equation*}
  for a unique monic $P_\alpha \in K[X]$. This $P_\alpha$ is called the minimal polynomial\index{minimal polynomial} of $\alpha$ over K. 
\end{definition}
Note: $\text{deg} P_\alpha$ is minimal amongst $\text{deg} P$ for $P \neq 0, P \in I_\alpha$.

\subparagraph{Example}
\begin{itemize}
\item The minimal polynomial of $\alpha = \sqrt{2}$ over $\mathbb{Q}$
  \begin{equation*}
    P_\alpha = X^2 -2 \in \mathbb{Q}[X]
  \end{equation*}
\item The minimal polynomia of $\alpha = \sqrt{2}$ over $\mathbb{R}$ is 
  \begin{equation*}
    P_\alpha = X-\sqrt{2} \in \mathbb{R}[X]
  \end{equation*}
\item The minimal polynomial of $\alpha = \sqrt[3]{2}$ over $\mathbb{Q}$ is
  \begin{equation*}
    P_\alpha = X^3 - 2 \in \mathbb{Q}[X]
  \end{equation*}
\end{itemize}

Consider $f_\alpha : \mathbb{Q} \rightarrow \mathbb{C}$, and let $\mathbb{Q}(\alpha) = \text{Im} f_\alpha = \{ a + b\alpha + c\alpha^2 / a, b, c \in \mathbb{Q}\} \subseteq \mathbb{C}$. This is a field , as it is a ring and every non zero element is invertible. 

e.g. find the inverse of $(1+\alpha)$, use:
\begin{equation*}
  (1+\alpha)(1-\alpha+\alpha^2) = 1+\alpha^3 = 3
\end{equation*}
hence we have that
\begin{equation*}
  (1+\alpha)^{-1} = \frac{1}{3}(1-\alpha+\alpha^2)
\end{equation*}

\section{Simple extensions}
Note: the intersection of subfields is a subfield. However, the union of subfields are not subfields in general.

\begin{definition} \label{def:6}
  Let $L/K$ be an extension and $\alpha \in L$. We denote by $K(\alpha)$ the intersection of all subfields of L, containing K and $\alpha$, i.e. the minimal such subfield. Then $K(\alpha)/K$ is called the extension generated\index{generated (extension} by $\alpha$. We say that $L/K$ is simple if it is generated by some $\alpha \in L$.
\end{definition}

\begin{proposition} \label{prop:7}
  Let $L/K$ be an extension and $\alpha \in L$. Then, we have that:
  \begin{enumerate}
  \item Its minimal polynomial is $P_\alpha$ over K is irreducible over $K[X]$.
  \item $\text{Im} f_\alpha = K(\alpha)$ and $[K(\alpha) : K] = \deg P_\alpha$. In particular, $K(\alpha)/K$ is finite.
  \end{enumerate}
\end{proposition}

\begin{proof} 
  \begin{enumerate}
  \item If $P_\alpha(X) = P(X)Q(X)$, then we have: $P(\alpha)Q(\alpha) = P_\alpha(\alpha) = 0$, hence $P(\alpha) = 0$ or $Q(\alpha) = 0$. Say $P(\alpha) = 0$, then $P \in I_\alpha$, i.e. $P_\alpha \mid P$, hence $Q$ is a unit $K[X]$.

  \item $\text{Im} f_\alpha$ is a subfield of $L$ 

It is certainly a ring as the image of a ring homomorphism. Every $x \in \text{Im}$ is of the form $P(\alpha)$ for some $P \in K[X]$. If $x \neq 0$, then we have that $P \not\in I_\alpha$, i.e. $P$ is not divisible by $P_\alpha$. Hence, $\exists Q \in K[X], with PQ \equiv 1 \mod{P_\alpha}$, therefore $P(\alpha)^{-1} = Q(\alpha) \in \text{Im} f_\alpha$.

  \item $\text{Im} f_\alpha = K(\alpha)$

As $\text{Im} f_\alpha$ is a subfiel of $L$ containing $K$ and $\alpha$, and every such field must contain $\text{Im} f_\alpha$. We have $\text{Im} f_\alpha = K(\alpha)$.

  \item If $\deg P_\alpha = n$, then $\{1, \alpha, \alpha^2, \ldots, \alpha^{n-1}\}$ gives a basis of $K(\alpha)$ as a vector space over K.

For every $x = P(\alpha) \in \text{Im} f_\alpha$, there exits $Q, E \in K[X]$ with $P = P_\alpha\cdot{}Q + R$ and $\deg R < n$ hence $x = P(\alpha) = R(\alpha)$ is a K-linear combination of $\{1, \alpha, \alpha^2, \ldots, \alpha^{n-1}\}$. If $R(\alpha) = 0$, with $\deg R < n$, then $P_\alpha \mid R$ by definition of the minimal polynomial, hence $R = 0$.
  \end{enumerate}
\end{proof}

\subparagraph{Remarks}
\begin{enumerate}
\item Different elements can generate the same field, i.e. we can have $K(\alpha) = K(\alpha')$ with $\alpha \neq \alpha'$, e.g.
  \begin{equation*}
    \mathbb{Q}(1+\sqrt{2}) = \mathbb{Q}(\sqrt{2})
  \end{equation*}

\item By \autoref{prop:4} and \autoref{prop:7} for an extension $L/K$ and $\alpha \in L$, then we have:
  \begin{equation*}
    \alpha \text{algebraic over K} \Leftrightarrow K(\alpha)/K \quad \text{is finite} 
  \end{equation*}

\item If $K \subseteq L \subseteq F$ and $\alpha \in F$, then $K[X] \subseteq L[X]$ implies:
  \begin{enumerate}
  \item if $\alpha$ algebraic over K, then $\alpha$ algebraic over L. Note that the converse holds when $L/K$ is finite.
  \item its minimal polynomial $Q_\alpha$ over L divides its minimal polynomial $P_\alpha$ over K.
  \end{enumerate} 

\item Related question:

We have that $\sqrt{2}$, $sqrt[3]{2}$ algebraic over $\mathbb{Q}$, does this imply $\sqrt{2} + \sqrt[3]{2}$ alebraic over $\mathbb{Q}$? Answer is yes, finding appropriate polynommail is left as an exercise for the reader.
\end{enumerate}

\section{Finite extensions}
\subparagraph{Note} if $[L : K] = 1$, then $L = K$ (as we have a 1-dimensional K-vector space).

\begin{proposition}[Tower Law]\label{prop:8}
  Let $K \subseteq L \subseteq F$, then if $L/K$, $F/L$ are finite extensions then so is $F/K$, and we have
  \begin{equation*}
    [F : K] = [F:L][L:K]
  \end{equation*}
\end{proposition}

\begin{proof}
  Let $\{a_1, \ldots, a_n \} \subseteq L$ be a basis of $L/K$ (i.e a basos of L as a vector space over K).
Let $\{ b_1, \ldots, b_m \}$ be a basis of $F/L$. Then every $x \in F$ is written as:
\begin{equation*}
  x = x_1b_1 + \ldots + x_mb_m
\end{equation*}
with $x_j \in L$ for $1 \leq j \leq m$, and each $x_j \in L$ is written as:
\begin{equation*}
  x_j = x_{1j}a_1 + \ldots + x_{nj}a_n
\end{equation*}
with $x_{ij} \in K$ for each $1 \leq i \leq n$, $1 \leq j \leq m$. Hence, we have that:
\begin{equation*}
  x = \sum_j\left(\sum_ix_{ij}a_i\right)b_j = \sum_{i, j}x_{ij}a_ib_j
\end{equation*}
If $x = \sum_{i, j}x_{ij}a_ib_j = 0$, then $\sum_j x_{ij}a_i = 0$ for all $j$ by the L-linear independence of $b_j$, therefore $x_{ij} = 0$ by K-linear independence of the $a_i$.

Thus, we have that the following is a basis of $F/K$:
\begin{equation*}
  \{ a_ib_j \mid 1 \leq i \leq n, 1 \leq j \leq m \}
\end{equation*}
\end{proof}






%%% Local Variables: 
%%% mode: latex
%%% TeX-master: "Galois"
%%% End: 
