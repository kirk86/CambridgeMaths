\begin{definition}\label{def:9}
  let $L/K$ be an extension and $\alpha_1, \ldots, \alpha_n \in L$. 

  We denote by $K(\alpha_1, \ldots ,\alpha_n)$ the intersection of all subfields of L containing these elements, i.e. K and $\alpha_1$, \ldots, $\alpha_n$. It is the minimal such subfield.

  We call $K(\alpha_1, \ldots, \alpha_n)/K$ an extension generated by $\alpha_1$, \ldots, $\alpha_n$ over K. The order of $\alpha_1$, \ldots, $\alpha_n$ is irrelevant, and we have that

  \begin{equation*}
    K(\alpha_1, \ldots, \alpha_n) = K(\alpha_1, \ldots, \alpha_{n-1})(\alpha_1)
  \end{equation*}
\end{definition}

\begin{proposition}\label{prop:10}
  \begin{enumerate}
  \item If $L/K$ is an extension and $\alpha_1, \ldots, \alpha_n \in L$ are algebraic over K, then $K(\alpha_1, \ldots, \alpha_n)$ is finite over K
  \item Conversely, every finite extensiion $L/K$ is generated by finitely many elements, i.e. $\exists \alpha_1, \ldots, \alpha_n \in L$ such that $L = K(\alpha_1, \ldots, \alpha_n)$.
  \end{enumerate}
\end{proposition}

\begin{proof}
  As $\alpha_n$ is algebraic over K, it is a fortiori algebraic over $K(\alpha_1, \ldots, \alpha_{n-1})$, hence $K(\alpha_1, \ldots, \alpha_n) = K(\alpha_1, \ldots, \alpha_{n-1})(\alpha_n)$ is finite over $K(\alpha_1, \ldots, \alpha_{n-1}$ by \ref{prop:7}. The result then follows by induction and tower law.

Conversely, take a basis $\{e_1, \ldots, e_n\}$ of $L/K$, then $K(e_1, \ldots, e_n) = L$ as every $x \in L$ is a K-linear combination. 
\end{proof}

\subparagraph{Example}
The minimal polynomial of $\sqrt[3]{2}$ over $\mathbb{Q}(\sqrt{2}]$ is $X^3 -2$, as it is irreducible in $\mathbb{Q}(\sqrt{2})[X]$ (else, its roots in $\mathbb{Q}(\sqrt{2})$ would generate a field of degree 3 over $\mathbb{Q}$).
\begin{figure}[H]
  \centering
  \begin{tikzpicture}
    \node (1) {$\mathbb{Q}(\sqrt{2})(\sqrt[3]{2}) = \mathbb{Q}(\sqrt{2}, \sqrt[3]{2})$};
    \node (2) [below = of 1] {$\mathbb{Q}(\sqrt{2})$};
    \node (3) [below = of 2] {$\mathbb{Q}$};

    \path 
    (1) edge node[right] {2} (2)
    (2) edge node[right] {2} (3);
  \end{tikzpicture}
\end{figure}
thus we have that $[\mathbb{Q}(\sqrt{2}, \sqrt[3]{2}) : \mathbb{Q}(\sqrt{2})] = 3$, and hence $[\mathbb{Q}(\sqrt{2}, \sqrt[3]{2}) : \mathbb{Q}] = 6$. The minimal polynomial of $\sqrt{2} + \sqrt[3]{2}$ has degree dividing 6.

\subparagraph{Example}

The fields of the form $K(\sqrt{a})$ and $K(\sqrt{a}, \sqrt{b})$ for non-square elements $a, b \in K$ are called quadratic\index{quadratic (extension)} and biquadratic\index{biquadratic (extension}) extensions of K.
\begin{figure}[H]
  \centering
\begin{tikzpicture}
  \node (1) {$\mathbb{Q}(\sqrt{2}, \sqrt{-1})$};
  \node (2) [below left = of 1] {$\mathbb{Q}(\sqrt{2})$};
  \node (3) [below = of 1] {$\mathbb{Q}(\sqrt{-1})$};
  \node (4) [below right = of 1] {$\mathbb{Q}(\sqrt{-2})$};
  \node (5) [below = of 3] {$\mathbb{Q}$};

  \path
  (1) edge node[above left] {2} (2)
  (1) edge node[auto] {2} (3)  
  (1) edge node[auto] {2} (4)
  (2) edge node[below left] {2} (5)
  (3) edge node[auto] {2} (5)
  (4) edge node[auto] {2} (5);
\end{tikzpicture}
\end{figure}
hence $\mathbb{Q}(\sqrt{2}, \sqrt{-1})$ is a finite extension of degree 4, with basis for example $\{1, \sqrt{2}, \sqrt{-1}, \sqrt{-2}\}$.

\subparagraph{Remark}
Finite extensions are algebraic, but there exists infinite algebraic extensions.

\section{K-homomorphisms}

\begin{lemma}\label{lemma:11}
  If L is a field and $\tau : L \rightarrow L'$ be any ring homomorphism (where $L'$ is not a zero ring), then $\tau$ is injective.
\end{lemma}

\begin{proof}
  As $\ker \tau$ is an ideal of L not containing $1 \in L$ (as $\tau(1) = 1$), we have that $\ker \tau = \{0\}$.
\end{proof}

\begin{definition}\label{def:12}
  Let $L/K$, $L'/K$ be two etensions of K. A K-homomorphism\index{K-homomorphism} is a ring homomorphism from L to L' such that $\restr{\tau}{K} = \text{id}$. 

  The set of all K-homomorphisms is denoted by $\text{Hom}_k\left(L, L'\right)$.
\end{definition}

\subparagraph{Note}

All K-homomorphisms are injective. They are sometimes called embedding, and they are K-linear.

We are mainly interested in the set $\homset{L}{\mathbb{C}}$ when $K \subseteq L \subset \mathbb{C}$.

\subparagraph{Example}

\begin{description}
\item[$\mathbb{C}/\mathbb{R}$:] The $\mathbb{R}$-homomorphisms $\tau : \mathbb{C} \rightarrow \mathbb{C}$ are either:
  \begin{equation*}
    z \mapsto z \quad \text{or} \quad z \mapsto \overline{z}
  \end{equation*}
\item [$\mathbb{Q}(\sqrt{2})/\mathbb{Q}$:] The $\mathbb{Q}$-homomorphisms of $\mathbb{Q}(\sqrt{2} \rightarrow \mathbb{C}$. There are two, which verify:
  \begin{IEEEeqnarray*}{rCl}
    \tau_1 &:& \sqrt{2} \mapsto \sqrt{2} \\
    \tau_2 &:& \sqrt{2} \mapsto \sqrt{-2}
  \end{IEEEeqnarray*}
Note that since it is a ring homomorphism, it must send a root of P to a root of $P \in K[X]$.
\item[$\mathbb{Q}(\protect{\sqrt[3]{2}})/\mathbb{Q}$:] Let $\zeta^3=1$, $\zeta \neq 1$, then we have three $\mathbb{Q}$-homomorphisms:
  \begin{IEEEeqnarray*}{rCl}
    \tau_1 &:& \sqrt[3]{2} \mapsto \sqrt[3]{2} \\
    \tau_2 &:& \sqrt[3]{2} \mapsto \zeta\sqrt[3]{2} \\
    \tau_3 &:& \sqrt[3]{2} \mapsto \zeta^2\sqrt[3]{2}
  \end{IEEEeqnarray*}
in this case, the images $\text{Im} \tau_i$ are different subfields of $\mathbb{C}$.
\end{description}

\begin{definition}\label{def:13}
  \begin{enumerate}
  \item For a non-zero $P \in K[X]$ and an extension $L/K$, we denote by $\text{Root}_P(L)$ the set of all roots of P in L.
  \item Let $\alpha \in L$ be algebraic over K. A root of its minimal polynomial $P_\alpha$ over K in L, i.e. an element of $\text{Root}_{P_\alpha}(L)$, is called a conjugate\index{conjugate} of $\alpha$ in L over K.
  \end{enumerate}
\end{definition}

\begin{proposition}[Roots and Homomorphisms I]\label{prop:14}
  Let $F/K$, $E/K$ be two extensions of K and $\alpha \in F$ be algebraic over K. Then, we have a bijection:
  \begin{equation*}
    \homset[K]{K(\alpha)}{E} \rightarrow \rootset[P_\alpha]{E}
  \end{equation*}

In particular, we have that:
\begin{equation*}
  \cardinality{\homset[K]{K_\alpha}{E}} \leq [K(\alpha) : K]
\end{equation*}
\end{proposition}

\begin{proof}
  \begin{enumerate}
  \item We have a well-defined map:
    As $P_\alpha(\alpha) = 0$, and $\tau$ is a ring homomorphism such that $\restr{\tau}{K} = \text{id}$, we have that:
    \begin{equation*}
      P_\alpha(\tau(\alpha)) = \tau{}P_\alpha(\alpha) = 0
    \end{equation*}
(every K-homomorphism sends $\alpha$ to one of its conjugates in K)

\item It is injective:
All elements in $K(\alpha)$ are polynomials in $\alpha$ with coefficients in K, and $\restr{\tau}{K} = \text{id}$, the map is determined by $\tau(\alpha) \in E$.

\item It is surjective:
let $\beta \in \rootset[P_\alpha]{E}$, we define $\tau : K(\alpha) \rightarrow E$ satisfying $\tau(\alpha) = \beta$. Every $x \in K$ is written $x = P(\alpha)$ with $P \in K[X]$ and P is unique up to adding multiples of of $P_\alpha$ (i.e. any other choice of P is of the form $P + P_{\alpha}Q$), and $P_\alpha(\beta) = 0$ hence $P(\beta) \in E$ is well-defined. Hence let $\tau(x) = P(\beta)$, i.e.
\begin{IEEEeqnarray*}{rCl}
  \tau : K(\alpha) & \rightarrow & E \\
         P(\alpha) & \mapsto & P(\beta)
\end{IEEEeqnarray*}
this is a ring homomorphism, and $\restr{\tau}{K} = \text{id}$.

\item $\cardinality{\homset[K]{K(\alpha)}{E}} = \cardinality{\rootset[P_\alpha]{E}} \leq \deg P_\alpha = [K(\alpha) : K]$.
  \end{enumerate}
\end{proof}


%%% Local Variables: 
%%% mode: latex
%%% TeX-master: "Galois"
%%% End: 