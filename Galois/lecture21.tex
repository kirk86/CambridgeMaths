
\chapter{Modern Galois Theory (Linear Algebra approach)}
\label{chap:3}

Here we rebuild Galois theory using more linear algebra, without ever mentioning polynomials. In this context the principal element theorem is avoided, we only assume the following: \autoref{def:1}, \autoref{def:2}, \autoref{prop:8}, \autoref{def:12}, and from Linear Algebra, \autoref{lemma:11}, \autoref{lemma:22}.

\section{Dedekind's and Artin's lemma}
\label{sec:3.1}

\begin{lemma}
  \label{lemma:86}
  Let $V$ be a finite dimensional vector space over $K$ and $E/K$ an extension. Let $\homset[\text{K-v.s.}]{V}{E}$ be the set of all $K$-linear maps $V \rightarrow E$, and define the addition and $E$-action by:
\[
(\rho + \rho')(x) = \rho(x) + \rho'(x), \ (\alpha\rho)(x)=\alpha\cdot{}\rho(x)
\]
Then it is an $E$-vector space with $\dim_E (\homset[K-\text{v.s.}]{V}{E}) = \dim_K V$.
\end{lemma}

\begin{proof}
  It satisfies the axioms of an $E$-vector space. Let $\{ e_1, \ldots, e_n \}$ be a basis of $V$. If we define $\rho_i \in \homset[K-\text{v.s.}]{V}{E}$ by $\rho_i(a_1e_1 + \cdots{} + a_ne_n) = a_i$, then every $\rho$ is uniquely written as $\rho = \rho(e_1)\rho_1 + \cdots{} + \rho(e_n)\rho_n$ (as we have that $\rho(x) = \rho(\sum_i a_ie_i) = \sum_i a_i\rho(e_i) = \sum_i \rho_i(x)\rho(e_i) \in E$), hence $\{ p_1, \ldots, p_n \}$ is a basis of $\homset[K-\text{v.s.}]{V}{E}$ as an $E$-v.s.
\end{proof}

\begin{proposition}[Dedekind's lemma]
  \label{prop:87}
  \index{Dedekind's lemma}
  Let $F/K$ be a finite extension. Then for any extension $E/K$, the subset $\homset{F}{E}$ of the $E$-vector space $\homset[K-\text{v.s.}]{V}{E}$ is $E$-linearly independent. In particular, $\card{\homset{F}{E}} \leq [F : K]$ (by \autoref{lemma:86}).
\end{proposition}

\paragraph{Remark}

We proved this inequality in \autoref{prop:76}.

\begin{proof}
  We prove that any finite subset $\{ \rho_1, \ldots, \rho_k \}$ of $\homset{F}{E}$ is $E$-linearly independent, by induction on $k$. Let the following $E$-linear relation hold: 
\[
a_1\rho_1 + \cdots + a_k\rho_k = 0 \tag{*} \label{prop:87:asterisk}
\] If $k = 1$, then $\rho_1 \neq 0$ hence $a_1 = 0$, hence the claim holds. Let $k > 1$, for any $x, y \in F$, we have $a_1\rho_1(x)\rho_1(y) + \cdots + a_k\rho_k(x)\rho_k(y) = a_1\rho_1(xy) + \cdots + a_k\rho_k(xy) = 0$. As $y$ is arbitrary, we have equality as a $K$-linear map (i.e. in $\homset[K-\text{v.s.}]{F}{E}$):
\[
a_1\rho_1(x)\rho_1 + \cdots + a_k\rho_k(x)\rho_k = 0
\]
multiply \eqref{prop:87:asterisk} by $p_k(x)$ gives $a_1p_k(x)p_1 + \cdots + a_kp_k(x)p_k = 0$, so subtracting we obtain:
\[
a_1(\rho_1(x)-\rho_k(x))\rho_1 + \cdots + a_{k-1}(\rho_{k-1}(x)-\rho_k(x)) = 0
\]
Then all coefficients are 0 by induction hypothesis, and as $x$ is arbitrary, we have $a_i(\rho_i -\rho_k) = 0$, ($1 \leq i \leq k-1$). If $a_i \neq 0$, then multiplying by $a_i^{-1}$ gives $\rho_i = \rho_k$. Now the case $k = 1$ gives $a_k = 0$.
\end{proof}

This implies $\card{\autset[K]{F}} \leq [F : K]$ for finite $L/K$ (\autoref{lemma:22}). Now recall \autoref{def:23} and \autoref{lemma:32}. Then the first part of \autoref{prop:33} (that $F/K$ Galois implies $F^G = K$) follows. We prove the second part next.

\begin{proposition}[Artin's lemma]
  \label{prop:88}
  Let $F/K$ be any extension. If $G$ is a finite subgroup of $\autset[K]{F}$, then $F/F^G$ is Galois and $\Gal{F/F^G} = G$.
\end{proposition}

\begin{proof}
  Let $G = \{ \rho_1, \ldots, \rho_n\}$ (with $\rho_1 = \text{id}$), and write $\bm{\rho}(x) = \left( \rho_1(x), \rho_2(x), \ldots, \rho_n(x) \right) \in F^n$ for $x \in F$. For $\bm{x} = (x_1, \ldots, x_n) \in F^n$, and $\rho \in G$; write $\rho(\bm{x}) = \big( \rho(x_1), \ldots, \rho(x_n) \big)$. Then $\rho(a\bm{x}) = \rho(a)\rho(\bm{x})$, since $\rho$ is a ring homomorphism, and the components of $\rho(\bm{\rho}(x)) = (\rho\rho_1(x), \ldots, \rho\rho_n(x)) \in F^n$ are a permutation of those $\bm{\rho}(x)$. Hence if we have 
 \[
 a_1\bm{\rho}(x_1) + \cdots + a_k\bm{\rho}(x_k) = 0 \tag{*} \label{prop:88:asterisk}
 \]
 for $a_1, \ldots, a_k \in F$, then by applying $\rho$ to \eqref{prop:88:asterisk} we get:
 \[
 \rho(a_1)\bm{\rho}(x_1) + \cdots + \rho(a_k)\bm{\rho}(x_k) = 0 \tag{$\star$} \label{prop:88:star}
 \]
Now we prove that if $\{x_1, \ldots, x_n\}$ is an $F^G$-linearly independent subset of $F$, then $\{ \bm{\rho}(x_1), \ldots, \bm{\rho}(x_n) \}$ is $F$-linearly independent in $F^n$, which implies $k \leq n  = \card{G}$. Use induction on $k$. If $k = 1$, then $\bm{\rho}(x_i) \neq 0$, since $x_1 \neq 0$, hence the claim holds. Assume an $F$-linear relation \eqref{prop:88:asterisk}. Replacing $a_i$ by $a_i/a_k$ when $a_k \neq 0$, we can assume $a_k + 0$ or $1$. Then as $\rho(a_k) = a_k$ for all $\rho \in G$, \eqref{prop:88:asterisk} - \eqref{prop:88:star} gives:
\[
(a_1 - \rho(a_1))\bm{\rho}(x_1) + \cdots + (a_{k-1} - \rho(a_{k-1}))\bm{\rho}(x_{k-1}) = 0
\]
The induction hypothesis shows that all coefficients are 0, and since $\rho$ was arbitrary, we have $a_i \in F^G$. Now the first component of \eqref{prop:88:asterisk} needs $a_1x_1 + \cdots + a_kx_k = 0$ (as $\rho_1 = \text{id}$), and the $F^G$-linear independence of $\{ x_1, \ldots, x_k \}$ implies all $a_i$ are 0. Thus $F/F^G$ is finite with $[F : F^G] \leq n = \card{G}$. Since we had $\card{G} \leq \card{\autset[F^G]{F}} \leq [F : F^G]$ already, $\card{\autset[F^G]{F}} = [F : F^G]$ and $G = \Gal{F/F^G}$.
\end{proof}

\section{Towers of extensions}
\label{sec:3.2}

We almost reproved the fundamental theorem of Galois (\autoref{thm:34}). Remains to prove that : $K \subseteq L \subseteq F$, $F/K$ Galois implies $F/L$ Galois. To deal with the towers, we generalize the notion of extensions.

\begin{definition}
  \label{def:89}
  Let $K$ be a field. We call a pair $(F, \tau)$ of a field $F$ and a ring homomorphism $\tau : K \rightarrow F$ an extension\index{extension} of $K$, and denote it by $F_\tau$. 

A morphism\index{morphism} $\rho : F_\tau \rightarrow F'_{\tau'}$ of extensions is a ring homomorphism $\rho : F \rightarrow F'$ such that $\rho\tau = \tau'$, i.e. the following diagram commutes.
  \begin{figure}[H]
    \centering
    \begin{tikzpicture}
      \node (1) {$K$};
      \node (2) [below = of 1] {$F$};
      \node (3) [right = of 2] {$F'$};

      \path 
        (1) edge[->] node[left] {$\tau$} (2)
        (2) edge[->] node[below] {$\rho$} (3)
        (1) edge[->] node[above right] {$\tau'$} (3);
    \end{tikzpicture}
  \end{figure}
We denote the set of all morphisms from $F_\tau$ to $F'_{\tau'}$ by $\homset[]{F_\tau}{F'_{\tau'}}$. If $\rho$ is bijective, then $\rho^{-1}$ is also a morphism, and we call $\rho$ an isomorphism. An automorphism of $F_\tau$ is an isomorphism from $F_\tau$ to itself, and $\autset[]{F_\tau}$ is the group of all automorphisms of $F_\tau$.
\end{definition}

\paragraph{Remark}

$K$ and $\tau(K)$ are isomorphic (\autoref{lemma:11}, and $F/\tau(K)$ is an extension in our previous definition.

%%% Local Variables: 
%%% mode: latex
%%% TeX-master: "Galois"
%%% End: 
