
\section{Infinite extensions etc.}

\subsection{What was it all about}

Let $K$ be a field. Then, the Galois theory over $K$ is the theory of fields that are finite dimensional vector spaces over $K$ (finite extensions) and the morphisms (i.e. $K$-homomorphisms) between them.

However, from the point of view of Category theory, the set of morphisms control the objects. For example, by the Fundamental theorem of Galois, the Galois groups control the fields.

(insert table 1)

Principle of algebraic geometry (over any ring $K$). 
\begin{enumerate}
\item Any ring A, finitely generated over K, is a quotient ring of a polynomial ring $A = K[X_1, \dotsc, X_n]/I$ with an ideal $I$. Here $I = \big(f_1, \dotsc, f_n\big)$ and $f_i$ are the ``equations'' in $X_1$, \ldots, $X_n$ over $K$.
\item A ``solution'' of these equations in a ring $E$ is a ring homomorphism $A \rightarrow E$. For example, a solution $x, y \in \mathbb{Q}$ of $x^n + y^n = 1$ (for $n > 2$, $xy = 0$, Fermat's last theorem) is equivalent to a ring homomorphism $\mathbb{Q}[X, Y]/(X^n + Y^n -1) \rightarrow \mathbb{Q}$, with $X \mapsto x$, $Y \mapsto y$.
\end{enumerate}

\subsection{Subfields of algebraic closures, and the absolute Galois groups}

\begin{definition}
  \label{def:102}
  Let $E/K$ be an extension, and $F$, $F'$ be subfields of $E$ containing $K$. Their composite field $FF'$ is the intersection of all subfields of $E$ containing $F$, $F'$, i.e. the minimal such subfield.
\end{definition}

If $F = K(\alpha_1, \dotsc, \alpha_n$ and $F' = K(\beta_1, \dotsc, \beta_m)$ then $FF' = K(\alpha_1, \dotsc, \alpha_n, \beta_1, \dotsc \beta_m)$. So if $F$, $F'$ are finite then so is $FF'/K$.

\begin{lemma}
  \label{lemma:103}
  Let $F$, $F'$, $E$ be as above, with $F/K$, $F'/K$ finite.
  \begin{enumerate}
  \item If $F/K$, $F'/K$ are separable (resp. Galois, soluble, abelian), then so is $FF'/K$.
  \item Let $E = \overline{K}$ be an algebraic closure of $K$. Let $K^{\text{Sep}}$ (resp. $K^{\text{sol}}$, $K^{\text{ab}}$, $K^{\text{cyc}}$) be the union of all finite (separable, soluble, abelian, cyclotomic) extensions of $F$ inside $\overline{K}$. Then they are fields, hence algebraic extensions. The field $K^{\text{Sep}}$, called a separable closure pf $K$, is equal to the union of all finite Galois extensions of $K$ inside $\overline{K}$.
  \end{enumerate}
\end{lemma}

\begin{proof}
  \begin{enumerate}
  \item \autoref{thm:77} solves the separable and Galois cases. Now $\Gal{FF'/K} \ni \sigma \mapsto \left(\restr{\sigma}{F}, \restr{\sigma}{F'}\right) \in \Gal{F/K}\times\Gal{F'/K}$ is injective. The soluble and abelian cases follow.

  \item For each family of subdields, any two members are contained in a larger member by i), so the union is a field (for cyclotomic case, note that $K(\unityroots{N}), K(\unityroots{N'}) \subseteq K(\unityroots{NN'})$). Any finite separable extension $F = K(\alpha_1, \dotsc, \alpha_n)$ is contained in the splitting field (inside $\overline{K}$) of the product of minimal polynomials of $\alpha_i$ over $K$ which is Galois (\autoref{prop:79}).
  \end{enumerate}
\end{proof}

\begin{definition}
  A algebraic extension $F/K$ (not necessarily finite) is called a Galois extension if it is a union of finite Galois extensions. Then $\Gal{F/K} = \autset[K]{F}$ is called its Galois group. The group $G_K = \Gal{K^{\text{sep}}/K}$ is well-defined up to isomorphism, is called the absolute Galois group\index{absolute Galois group} of $K$.
\end{definition}

By \autoref{lemma:103}, for any field $K$, we have a sequence of Galois extensions: 
\[
K \subseteq K^{\text{cyc}} \subseteq K^{\text{ab}} \subseteq K^{\text{sol}} \subseteq K^{\text{sep}} \subseteq \overline{K}
\].
Every finite Galois extension has a $K$-isomorphic copy inside $K^{\text{sep}}$. If $F \subseteq K^{\text{sep}}$, then every element of $\Gal{F/K}$ can be extended to $K^{\text{sep}}$ using \autoref{prop:76}, hence $\sigma \mapsto \restr{\sigma}{F}$ gives a surjection $G_K \rightarrow \Gal{F/K}$.

\subsection{What next?}

\paragraph{Algebra}

By Abel and Galois (insolvability of quintics), we have that $K^{\text{sol}} \neq \overline{K}$ in general. However, little can be said about $G_K$ for general $K$.  On the contrary, $G_K$ knows a lot about each specific field $K$. 

\subparagraph{Example}

\begin{enumerate}
\item $K = \finitefield{p}$, we have that $\finitefield{p}^{\text{cyc}} = \dotsb = \overline{\finitefield{p}}$.
\item $K = \mathbb{Q}$. We know that $\mathbb{Q}^{\text{cyc}} = \mathbb{Q}^{\text{ab}}$ (Kronecker-Weber theorem). For every finite soluble group G, there is a Galois extension $F/\mathbb{Q}$ with $\Gal{F/\mathbb{Q}} = G$ (Shefarenial theorem). Is it true for arbitrary finite group (inverse Galois problem)?
\item $K/\mathbb{Q}$ finite (number fields). Class field theory describes $\Gal{K^{\text{ab}}/K}$. In some case $K^{\text{ab}}$ is understood via modular and elliptic functions (this is where Abel started).
\end{enumerate}

\paragraph{Geometry}

(insert figure 2)

\begin{tabular}{ll}
  $G_K$ & fundamental groups \\
  $\finitefield{p}$ has cyclic degree $n$ extension for each $n$ & (insert figure 3) $S^1 = \{ e^{i\theta} \mid \theta \in \mathbb{R} \}$. For each $n$, $\theta \mapsto n\theta$.
\end{tabular}