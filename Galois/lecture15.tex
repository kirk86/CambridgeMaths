\begin{theorem}
  \label{thm:65}
  \begin{enumerate}
  \item For each prime power $q = p^n$, there exists a finite field with $q$ elements, unique up to $\mathbb{F}_p$-ismorphism (i.e. field isomorphism). This field is denoted by $\mathbb{F}_q$.
  \item Let $m, n \geq 1$, and $q = p^n$, $q' = p^m$. Then $\mathbb{F}_{q'}$ contains $\mathbb{F}_q$ if and only if $q'$ is a power of $q$, i.e. $n \mid m$. If  $q' = q^d$, then $[\mathbb{F}_{q'} : \mathbb{F}_{q}] = d$.
  \end{enumerate}
\end{theorem}

\begin{proof}
  \begin{enumerate}
  \item Let K be a splitting field of $X^4-X$ over $\mathbb{F}_p$ (\autoref{prop:56}). By corollary \autoref{cor:62}, $X^q-X$ has $q$ distinct roots in K (as $p$ does not divide $q-1$). Then, the set of all roots $\{ x \in K \mid x^q = x\}$ is a subfield of K by lemma \autoref{lem:63} (as $a^q = a, b^q = a \Rightarrow (a+b)^q = a + b, (ab)^q = ab, \left(a^{-1}\right)^q = a^{-1}$). As K is a splitting field, i.e. generated by all roots, it is equal to this subfield, and $\card{K} = q$. By lemma \autoref{lem:59}, every field with $q$ elements is $\mathbb{F}_p$-isomorphic to this one, by uniqueness of splitting fields.

  \item If $q' = q^d$, then every root of $X^q-X$ is a root of $X^{q'}-X$. Since $x^q = x$ implies $x^{q'} = x$, hence $\mathbb{F}_{q'}$ contains $\mathbb{F}_{q}$. Conversely, if $\mathbb{F}_q \subseteq \mathbb{F}_{q'}$, then $\mathbb{F}_{q'}$ is a vector space over $\mathbb{F}_q$, and if $[\mathbb{F}_{q'} : \mathbb{F}_q] = d$, then $q' = \card{\mathbb{F}_{q'}} = \card{\mathbb{F}_q}^d = q^d$.
  \end{enumerate}
\end{proof}

\subparagraph{Remark}

Any field contains at most one copy of $\mathbb{F}_q$ by lemma \autoref{lemma:59}. This justifies the use of the canonical notation $\mathbb{F}_p$ even if the field is not canonically constructed, as it is usually unique in a given context.

\subparagraph{Example}

We have such a diagram for every $p$, but here, $p = 2$.

\begin{figure}[H]
  \centering
  \begin{tikzpicture}
    \node (2) {$\finitefield{2}$};
    \node (4) [above = of 2] {$\finitefield{4}$};
    \node (16) [above = of 4] {$\finitefield{16}$};
    \node (256) [above = of 16] {$\finitefield{256}$};

    \node (8) [above left = of 2] {$\finitefield{8}$};
    \node (64) [above = of 8] {$\finitefield{64}$};
    \node (4096) [above = of 64] {$\finitefield{4096}$};
    
    \node (512) [above left = of 8] {$\finitefield{512}$};

    \path
      (2) edge node[auto] {2} (4)
      (4) edge node[auto] {2} (16)
      (16) edge node[auto] {2} (256)
      (2) edge node[auto] {3} (8)
      (4) edge node[auto] {3} (64)
      (16) edge node[auto] {3} (4096)
      (8) edge node[auto] {2} (64)
      (64) edge node[auto] {2} (4096)
      (8) edge node[auto] {3} (512);
  \end{tikzpicture}
\end{figure}
The union of all these fields is $\overline{\finitefield{p}}$, and we have:
\begin{equation*}
  \finitefield{q} = \{ x \in \overline{\finitefield{q}} \mid x^q = x \} \text{ for } q = p^n
\end{equation*}

\begin{lemma}
  \label{lemma:66}
  Consider a finite extension $\finitefield{q^n}/\finitefield{q}$. Then the Frobenius map\index{Frobenius map} $\text{Fr}_q : x \mapsto x^q$ is a $\finitefield{q}$-automorphism of $\finitefield{q^n}$ with order $n$, i.e. 
  \begin{equation*}
    \{ \text{id}, \text{Fr}_q, \text{Fr}_q^2, \ldots, \text{Fr}_q^{n-1} \} \subseteq \autset[\finitefield{q}]{\finitefield{q^n}} \}
  \end{equation*}
\end{lemma}

\begin{proof}
  The map $\text{Fr}_q$ fixes all elements in $\finitefield{q}$ (the roots of $X^q-X$, hence is a $\finitefield{q}$-homomorphism. Being an injective map (by \autoref{lemma:11}) of a finite set $\finitefield{q^n}$ into itself, it is a $\finitefield{q}$-automorphism. Since $\forall x \in \finitefield{q^n}$, $x^{q^n}=x$, we have $\text{Fr}_q^n = \text{id}$ on $\finitefield{q^n}$, i.e. the order of $\text{Fr}_q$ in $\autset[\finitefield{q}]{\finitefield{q^n}}$ divides $n$. But, for each $m \mid n$, the elements fixed by $\text{Fr}_q^m$ are exactly the elements in $\finitefield{q^m}$, so the order is $n$.
\end{proof}

\subparagraph{Remark}

For a non-finite K with $\mchar K = p$, the Frobenius map is injective but not necessarily an automorphism (e.g. $\text{Fr} : \finitefield{p}(X) \rightarrow \finitefield{p}(X)$ has image $\finitefield{p}(X^p)$, where $\finitefield{p}(X)$ denotes the field of rational functions). 

\begin{lemma}
  \label{lemma:67}
  Let K be a field, and $K^{\times} = K \setminus \{ 0 \}$ be its multiplicative group. Then every finite subgroup $G$ of $K^{\times}$ is cyclic.
\end{lemma}

\begin{proof}
  Let $x \in G$ be an element with the maximal order, $n$ say. We show that for any $y \in G$, its order $m$ has to divide $n$. Suppose not, then there exists a $p$ prime such that $m = p^jm'$, $n = p^kn'$, and $j > k$ (with $p \nmid m', n'$).

  Let $z = x^{p^k}y^{m'}$, then we have:
  \begin{IEEEeqnarray*}{rCl}
    z^i = 1 &\Rightarrow& x^{{p^k}i} = y^{-im'} \\
    &\Rightarrow&
    \begin{cases}
      x^{p^jp^ki} = y^{-im} = 1 \Rightarrow n \mid p^{j+k}i \Rightarrow n' \mid i \\
      1 = x^{ni} = y^{-im'n'} \Rightarrow m \mid im'n' \Rightarrow p^j  \mid i
    \end{cases} \\
    &\Rightarrow& p^jn' \mid i
  \end{IEEEeqnarray*}
i.e. the order of $z$ is $p^jn' > n$, hence contradiction. Now $x^{in/m}$ ($ 1 \leq i \leq m$) are in distinct roots of $X^m -1$ in K, hence all the roots (lemma \ref{lemma:53}). As $y$ is a root, it is a power of $x$.
\end{proof}

\begin{theorem}
  \label{thm:68}
  Every finite extension $\finitefield{q'}/\finitefield{q}$ is simple and Galois with:
  \begin{equation*}
    \Gal{\finitefield{q^n}/\finitefield{q}} = \{ \text{id}, \text{Fr}, \text{Fr}_q^2, \ldots,  \text{Fr}_q^n \} \cong C_n
  \end{equation*}
\end{theorem}

\begin{proof}
  By \autoref{lemma:67}, $\finitefield{q^n}^\times = \finitefield{q^n} \setminus \{ 0 \}$ is cyclic, i.e. $\finitefield{q}^n  = \{ 0, 1, \zeta, \zeta^2, \ldots, \zeta^{q^n-2} \}$ for some $\zeta \in \finitefield{q^n}$, hence $\finitefield{q^n} = \finitefield{q}(\zeta)$ is simple. If $P_\zeta$ is the minimal polynomial of $\zeta$ over $\finitefield{q}$, then:
  \begin{equation*}
    \card{\autset[\finitefield{q}]{\finitefield{q^n}}} \leq \card{\homset[\finitefield{q}]{\finitefield{q^n}}{\finitefield{q^n}}} = \card{\rootset[P_\zeta]{\finitefield{q^n}}} \leq n
  \end{equation*}
by \autoref{prop:14}. Now \autoref{lemma:66} implies the rest.
\end{proof}

\subparagraph{Remark}

\autoref{thm:68} and \autoref{prop:14} imply that conjugates of $\zeta$ are:
\begin{equation*}
  \{ \text{Fr}_q^i(\zeta) = \zeta^{q^i} \mid 0 \leq i \leq n-1 \}
\end{equation*}
i.e.
\begin{equation*}
  P_\zeta = (X-\zeta)(X-\zeta^q)(X-\zeta^{q^i})\cdots(X-\zeta^{q^{n-1}})
\end{equation*}
with $\deg P_\zeta = [\finitefield{q^n} : \finitefield{q}]$.

\subparagraph{Example}

Not all generators $\zeta$ of the group $\finitefield{q^n}^\times$ are conjugate over $\finitefield{q}$. For example, in $\finitefield{2}[X]$:
\begin{equation*}
  X^16 - X = \underbrace{\underbrace{X(X+1)}_{\finitefield{2}}(X^2+X+1)}_{\finitefield{4}}\underbrace{(X^4+X^3+X^2+1)(X^4+X+1)(X^4+X^3+1)}_{\text{roots all generate } \finitefield{16}/\finitefield{2}}
\end{equation*}
whose roots are all elements in $\finitefield{2}$. The 8 last roots are generators of $\finitefield{16}^\times \cong C_{15} \cong C_3 \times C_5$ by \autoref{lemma:67}.


%%% Local Variables: 
%%% mode: latex
%%% TeX-master: "Galois"
%%% End: 
